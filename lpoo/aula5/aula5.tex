\documentclass[a4paper,10pt]{article}
\usepackage[utf8]{inputenc}
\usepackage[brazil]{babel}
\usepackage[T1]{fontenc}
\usepackage{longtable}

%opening
\title{Aula 5 - Atividade EAD - Conceitos de Herança}
\author{Augusto Cesar de Aquino Ribas}

\begin{document}

\maketitle
Trabalho para a disciplina de linguagem de programação orientada a objetos do professor Samuel Ferraz



\section{Introdução}

A herança reduz o tempo de desenvolvimento do programa.

A superclasse direta de uma subclasse(especificada pela palavra-chave extends na primeira linha de uma declaração de classe) é a superclasse a partir da qual a subclasse herda. A superclasse indireta de uma subclasse está dois ou mais níveis acima da hierarquia de classe dessa subclasse.

Em herança única, uma classe é derivada de uma superclasse direta. Na herança múltipla, uma classe é derivada de mais de uma superclasse direta. O java não suporta herança múltipla.

Uma subclasse é mais específica que sua superclasse e representa um grupo menor de objetos.

Cada objeto de uma subclasse também é um objeto da superclasse dessa classe. Entretanto, um objeto de superclasse não é um objeto de subclasse da sua classe.

Um relacionamento \emph{é um} representa herança. Em um relacionamento \emph{é um}, um objeto de uma subclasse também pode ser tratado como um objeto de sua superclasse.

Um relacionamento \emph{tem um} representa a composição. Em um relacionamento \emph{tem um}, um objeto de classe contém referencias a objetos de outras classes.

\section{Superclasses e subclasses}

Os relacionamentos de herança simples formam estruturas hierárquicas do tipo árvore - há uma superclasse em um relacionamento hierárquico com suas subclasses.

\section{Membros protected}

Os membros public de uma superclasse são acessíveis onde quer que o programa tenha uma referência a um objeto dessa superclasse ou para uma de suas subclasses.

Os membros de uma superclasse private só podem ser acessados diretamente a partir de dentro da declaração da superclasse.

Os membros protected de uma superclasse tem um nível intermediário de proteção entre acesso public e private, Eles podem ser acessados por membros da superclasse, por membros de suas subclasses e por membros de outras classes no mesmo pacote.

Os membros private de uma superclasse permanecem ocultos nas suas subclasses e só podem ser acessados por meio dos métodos publics ou protected herdados da superclasse.

Quando um método de subclasse sobrescrever um método de superclasse, o método de superclasse pode ser acessado a partir da subclasse se o nome de método de superclasse for precedido por super e um ponto separados (.).

\section{Relacionamento entre superclasses e subclasses}

Uma subclasse não pode acessar os membros private de sua superclasse - permitir isso violaria o encapsulamento da superclasse. Uma subclasse pode, porém, acessar os membros não private de sua superclasse.

Uma subclasse pode invocar um construtor da sua superclasse utilizando a palavra-chave super, seguida pelo conjunto de parenteses contendo os argumentos do construtor de superclasse. Isso deve aparecer como a primeira instrução no corpo do construtor da subclasse.

Um método de superclasse pode ser sobrescrito em uma subclasse para declarar uma implementação apropriada para a subclasse.

A  notação @Override indica que um método deve sobrescrever um método de superclasse. Quando o compilador encontrar um método declarado com @Override, ele comparará a assinatura do método com as assinaturas de método da superclasse. Se nao houver uma correspondência exata, o compilador emite uma mensagem de erro, como 'method does not override or implement a method from a supertype' (método não sobrescreve ou implementa um método a prtir de um supertipo).

O método toString não recebe nenhum argumento e retorna uma String. O método toString da classe Object normalmente é sobrescrito por uma classe.

Quando um objeto é enviado para saída utilizando o especificador de formato $\%s$, o método toString do objeto é chamado implicitmente para obter representação de String.

\section{Construtures em subclasses}

A primeira tarefa de qualquer construtor de subclasse é chamar o construtor de sua superclasse direta, explicita ou implicitmente, para assegurar que as variáveis de instância herdadas da superclasse são inicializadas.

\section{Engenharia de software com herança}

Declarar variáveis de instância private, ao fornecer métodos não private para manipular e realizar a validação, ajuda a impor uma boa engenharia de software.

\section{Classe Object}

Classe Object é a superclasse usada na linguagem Java, todos as classes são por padrão extenções dessa classe.

\section{Referência}

Paul Deitel e Harvey Deitel 2010 Java. Como Programar $8^a$ edição Pearson Prentice Hall 1143pp.
































\end{document}
