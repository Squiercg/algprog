\documentclass[a4paper,10pt]{article}
\usepackage[utf8]{inputenc}
\usepackage[brazil]{babel}
\usepackage[T1]{fontenc}

\usepackage[cm]{fullpage}
\usepackage{enumerate}

\author{Empreendedorismo - Augusto Cesar de Aquino Ribas}
\title{Exercício em sala - Capítulo 4}


\begin{document}

\maketitle

\section{Descreva os principais passos para a construção de um negócio de alta tecnologia. Quais os principais elementos ou fases? Disponha esses elementos graficamente em uma linha do tempo.}


\begin{itemize}
\item Desenvolver um produto de tecnologia. - É preciso ter algo a ser comercializado.
\item Estruturar o negócio. - É preciso ter uma estrutura, uma forma de como comercializar o produto.
\item Testar o mercado.  - O produto precisa ser útil para os clientes, e este tem que ter interesse no produto.
\item Começar a fase de start up (crescimento). - É preciso aumentar a empresa, e conseguir ser eficiente com um tamanho e mercado maior.
\end{itemize}


\section{O que é e para que serve um plano de negócios?}

Um plano de negócios é uma forma estruturada de planejar e documentar todo o ciclo de vida de um negócio.

\section{Quais informações um plano de negócios deve conter? Que perguntas precisam ser respondidas? Apresente informalmente.}

Comumente, no plano de negócio deve conter o tamanho do negócio, quantidade de produtos,
se a Empresa é nova ou já existente, o tipo de negócio (produto, serviço, comércio ou indústria?) e o objetivo (captação de investimentos ou planejamento interno?).

\begin{enumerate}[a)]

\item Qual é exatamente o público-alvo?
\item O negócio atende a uma necessidade do público-alvo que não está sendo bem atendida pelo mercado? Qual é essa necessidade?
\item Qual é exatamente o produto ou serviço oferecido?
\item Qual é a fonte de receita? Quais são os preços?
\item Como o produto oferecido satisfaz as necessidades do público-alvo?
\item Quanto vale, monetariamente, para o público-alvo, o produto ou serviço oferecido?
\item Qual é o mercado atual, em número de clientes e em volume fi nanceiro? Qual é o mercado futuro? O mercado está crescendo? Que mercado você já detém e que mercado planeja deter no futuro?
\item Qual é a ação da concorrência? Quais seus pontos fortes e fracos? Que parcela das necessidades do público-alvo é bem atendida pela concorrência?
\item Como o produto oferecido se diferencia dos concorrentes atuais?
\item Será possível manter essa diferenciação ao longo do tempo?
\item Como você pretende atingir todo o seu público-alvo?
\item Qual é a expectativa de vendas?
\item Que infraestrutura tecnológica será necessária para viabilizar o negócio
(linha de produção, equipamentos para desenvolvimento, assistência técnica etc.)?
\item Quais parcerias já foram celebradas, e quais devem ser buscadas?
\item Qual é a equipe de empreendedores? Que aptidões estão bem supridas? Que aptidões precisam ser buscadas?
\item Em que estágio está o desenvolvimento? Quais os próximos passos e o cronograma?
\item Quais são as receitas e despesas atuais? Quais serão as receitas e despesas ao longo dos próximos anos?
\item De quanto investimento o negócio precisa para se viabilizar? Em que momento exatamente esse investimento será necessário? No que esse dinheiro será investido (vendas, produção etc.)?
\item Que montante você está solicitando de um potencial investidor? O que você oferece em troca (percentual da empresa, taxa de retorno)?
\item Quais os principais riscos do negócio? Como eliminá-los ou minimizá-los?
\item Qual é a missão da empresa? Qual a sua visão estratégica? Quais seus pontos fortes e fracos? Qual a situação pretendida para o futuro?
\end{enumerate}


\section{Proponha um modelo, ou seja, a organização (os tópicos) para um plano de negócios.}

\begin{itemize}

\item Sumário Executivo.

\item Plano Detalhado.
\begin{itemize}
\item Conceito do negocio
\item Análise de mercado
\item Plano de marketing
\item Estrutura organizacional
\item Plano de implantação
\item Planejamento financeiro
\end{itemize}
\end{itemize}
\section{Quais informações financeiras um plano de negócios deve conter? Quais os possíveis objetivos dessas informações?}

O planejamento financeiro costuma mostrar a situação financeira atual (no caso de empresa já existente), a expectativa de movimentação financeira para os próximos anos (projeção de fluxo de caixa), os investimentos necessários para viabilizar o negócio, o planejamento das negociações com investidores, quando pertinente, e a projeção dos principais resultados: lucratividade, rentabilidade e retorno dos investimentos realizados.

Essas informações podem primeiro, expor a viabilidade de iniciar o negocio, caso seja necessário investimento, a possível quantidade de financiamento necessário e o tempo de retorno e rentabilidade do investimento, o que pode ajudar a captação desses recursos necessários. 


\section{O que quer dizer a frase: "o plano de negócios deve ser um documento em contínuo desenvolvimento"?}

O plano de negócios deve ser constantemente ajustado de acordo com o seu desenvolvimento, conforme novas informações são descobertas, ou necessidade de mudanças aparece.


\section{Explique por que um plano de negócios pode ser considerado como um instrumento de planejamento, um instrumento de integração de equipes e um instrumento para facilitar a captação de investimentos.}

Um plano de negócios é primeiramente um instrumento de planejamento. Uma forma para organizar e harmonizar os diferentes aspectos relevantes ao sucesso de um empreendimento. No fundo, o primeiro objetivo é evitar decepções, imprevistos, e aumentar as chances de sucesso. O plano de negócios pode servir também para facilitar a negociação com possíveis investidores e melhorar a comunicação e a integração da equipe que trabalha no empreendimento.

\end{document}
