\documentclass[a4paper,10pt]{article}
\usepackage[utf8]{inputenc}
\usepackage[brazil]{babel}
\usepackage[T1]{fontenc}

\usepackage[cm]{fullpage}

\author{Empreendedorismo - Augusto Cesar de Aquino Ribas}
\title{Exercício em sala - Capítulo 2}


\begin{document}

\maketitle

\section{Cite pelo menos três diferentes motivações para empreender.}

Podemos citar como motivação para empreender a necessidade de sobrevivência, quando se perde o emprego ou é preciso complementar a renda familiar, para investir quando se tem dinheiro ocioso e aparece uma opção mais rentável que aplicações de renda fixa e por amor a uma causa, quando se quer muito trabalhar em algum tipo de negócio.

\section{Um negócio que não dá lucro vale a pena? Cite uma situação em que vale a pena, e uma situação em que não vale a pena.}

Pode valer a pena se ele proporcionar renda suficiente para a sobrevivência, como um negócio familiar, mas é ruim para um investidor, se o negócio não da lucro, já que um rendimento melhor ou igual pode ser conseguido sem risco (renda fixa).


\section{Considerando exclusivamente o fator investimento, quais seriam os principais critérios para a escolha de um negócio?}


Primeiro ele precisa de uma rentabilidade maior que aplicações de renda fixa, que são mais seguras, além de ser um negócio profissionalizado, já que negócios mais amadores, familiares, por falta de estruturação, podem ser muito arriscados.


\section{Como seria possível aliar paixão e dinheiro em um empreendimento? Qual seria o caminho para escolher um negócio?}

Podemos procurar atividades que gostamos e então fazer uma lista de formas de ganhar dinheiro com ela, avaliando estas quando a possibilidade de se tornar um negócio e sua possível rentabilidade.


\end{document}
