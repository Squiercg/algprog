\documentclass[a4paper,10pt]{article}
\usepackage[utf8]{inputenc}
\usepackage[brazil]{babel}
\usepackage[T1]{fontenc}

\usepackage[cm]{fullpage}

\author{Empreendedorismo - Augusto Cesar de Aquino Ribas}
\title{Exercício em sala - Capítulo 3}


\begin{document}

\maketitle

\section{O que significa o termo valor agregado? Qual a relação entre desenvolvimento tecnológico e valor agregado?}

Valor agregado é o valor adicionado a uma quantidade de um produto, conforme ele é mais industrializado, sofre mais investimento. Desenvolvimento tecnológico é um tipo de investimento, que agrega valor agregado. 


\section{Por que o desenvolvimento tecnológico é uma vantagem competitiva duradoura? Cite um exemplo de vantagem competitiva menos duradoura do que o desenvolvimento tecnológico.}

Enquanto certas vantagens requerem apenas poder financeiro, como adicionar um brinde ao produto. Desenvolvimento tecnológico normalmente requer mão de obra especializada e tempo. Assim pode durar mais tempo que uma simples "promoção", que pode ser facilmente copiada pela concorrência.


\section{A inovação pode gerar melhorias em produtos já existentes. De que outras maneiras ela pode se manifestar?}

Inovação pode gerar produtos diferentes, abrindo novos nichos de mercado e mudando a forma como vivemos nossa vida ou consumimos.


\section{Por que a inovação tecnológica gera oportunidades de negócios? O que tem maior potencial para impacto no mercado: uma pequena inovação melhorando um produto já existente, ou uma revolução tecnológica? Explique por meio de exemplos.}

Pois ela pode criar novos mercados consumidores e paradigmas de consumo. Uma revolução tecnológica tem grande impacto, muita vezes alterando como vivemos o dia a dia, como é o exemplo dos smartfones e redes sociais, invenções do nosso século que hoje são quase impossíveis de serem evitadas pelas pessoas.


\section{O que é uma empresa de tecnologia? Caracterize a atuação do empreendedor tecnológico. Em que sentido o empreendedor tecnológico precisa "prever o futuro"? Por que ele precisa melhorar a vida das pessoas?}

É uma empresa que investe em inovação. O empreendedor tecnológico tentara abrir novos mercados de consumo, ao invés de simplesmente competir pelo que já existe e para tanto é preciso estar a par de possibilidades de negocio que hoje podem não ser uma realidade ainda. Mas inovação, tecnologia, só será aceita pelo mercado consumidor se realmente facilitar ou agregar a vida dos consumidores, ninguém quer comprar um computador novo se ele for muito mais complicado de utilizar e não aumentar em nada sua produtividade, ou facilitar sua vida.

\section{Qual a diferença entre compartilhar mercado e criar um novo mercado? Qual a diferença em relação à determinação de preços?}

Quando se tem um mercado compartilhado, é preciso saber basear seus preços dependendo do preço aplicado pela concorrência e o valor agregado de nosso produto, ou não conseguiremos vender e gerar renda para a empresa. Enquanto ao criar um novo mercado faz com que nos preocupemos apenas com o valor agregado de nosso produto e o acesso ao consumidor, ja que não existiram competidores para vender a um menor preço, temos um monopólio.


\section{O que significa a "Regra da volta à estaca zero"?}

Quando temos tecnologia disruptivas, aquelas que mudam drasticamente comportamentos dos consumidores, podemos dizer que voltamos a estaca zero, todas os empreendedores tem que trabalhar com essa nova tecnologia, e se basear nela para novos produtos, como por exemplo quando surgiu o cd, que fez com que o mercado de fitas cassetes sumisse, somente sobrando as empresas aderirem a nova tecnologia, mesmo que começando do zero.

\end{document}
