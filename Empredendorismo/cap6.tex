\documentclass[a4paper,10pt]{article}
\usepackage[utf8]{inputenc}
\usepackage[brazil]{babel}
\usepackage[T1]{fontenc}

\usepackage[cm]{fullpage}

\author{Empreendedorismo - Augusto Cesar de Aquino Ribas}
\title{Exercício em sala - Capítulo 6}


\begin{document}

\maketitle

\section{Qual a diferença entre lucro e rentabilidade?}

A quantidade que sobra quando se é retirada da receita bruta todo as depesas, chamamos de lucro. Já a rentabilidade é a razão do lucro dividido pelos investimentos totais, resultado na taxa de retorno do investimento.

\section{"Um dos objetivos de fazer uma projeção de fluxo de caixa é garantir que a liquidez nunca seja negativa." O que significa liquidez nesse caso? Como é possível garantir que a liquidez nunca será negativa? Explique a frase.}

Por exemplo, para meses onde a lucratividade for negativa, outros meses podem cobrir esse prejuízo se contabilizarmos o lucro num intervalo de tempo maior, assim prever o fluxo de caixa ao longo do ano pode garantir a liquidez.

\section{Em  uma projeção  de  fluxo  de  caixa o  termo  liquidez  tem um significado (discutido no Exercício 6.2). No contexto da análise de investimentos, o termo liquidez tem outro significado. O que significa liquidez no contexto da análise de investimentos}

É a facilidade de se converter o investimento em dinheiro vivo.

\section{Se você tivesse de estimar quanto vale uma empresa, ou se tivesse de concluir se um valor estimado está apropriado, que parâmetros financeiros buscaria saber? Por quê? Se para formar sua opinião você tivesse de escolher entre lucratividade e lucro líquido, qual desses escolheria? Por quê?}

Lucro líquido, rentabilidade e patrimônio para uma estimativa geral do valor da empresa.
Lucro líquido: pois esse é um valor independente de outras variáveis da empresa.

\section{Se você quisesse convencer um investidor a injetar dinheiro no seu negócio, o que ofereceria a ele? Que parâmetros financeiros você mostraria ou proporia ao investidor? Por quê?}

Ganho de capital, que é a taxa de retorno sobre o investimento dele. A partir deste parâmetro, o investidor pode calcular o risco da operação e projetar sua lucratividade.


\section{"Para um negócio valer a pena como investimento, a rentabilidade proporcionada precisa ser muito maior do que a da renda fixa." Você concorda com essa frase? Por quê?}

Não. A afirmação é verdadeira, porém não necessariamente uma regra, pois para qualquer empreendimento ou investimento existe o fator de risco ligado à liquidez e a rentabilidade do negócio, que pode ocasionar períodos com baixos ganhos, em troca de outros com lucros líquidos muito maiores. Ao se escolher apenas negócios com alta rentabilidade os riscos são menores, porém isso não inflige em altos lucros.

\section{Que critérios podem ser utilizados para analisar se um investimento é bom ou ruim? Cite critérios financeiros e critérios não-financeiros.}

Não financeiros: risco, a liquidez e o ônus de gestão.
Financeiros: lucro líquido, rentabilidade e patrimônio.

\section{Qual  a  diferença  entre  rentabilidade  e  taxa  de  retorno  do investimento?}

Rentabilidade é a razão do lucro líquido do período sobre o investimento total
O retorno sobre o investimento é a taxa que um investidor receberá de acordo com o valor inserido no negócio.

\section{Como posso calcular a lucratividade de um negócio? E como posso calcular a rentabilidade?}

A lucratividade é o percentual resultante da divisão do lucro líquido sobre a receita bruta. Rentabilidade é a razão entre o lucro líquido no período sobre o investimento total.

\section{Que elementos você apontaria como essenciais no planejamento financeiro de um novo negócio?}

 O investimento necessário e a projeção da rentabilidade e do retorno dos investimentos realizados, que viabilizaria esse investimento.

\section{Como  posso  determinar  o  investimento  necessário  para um novo negócio?}

Através da projeção de fluxo de caixa.

\end{document}
