\documentclass[a4paper,10pt]{article}
\usepackage[utf8]{inputenc}
\usepackage[brazil]{babel}
\usepackage[T1]{fontenc}

\usepackage[cm]{fullpage}

\author{Empreendedorismo - Augusto Cesar de Aquino Ribas}
\title{Exercício em sala - Capítulo 3}


\begin{document}

\maketitle

\section{O que pode ser protegido por uma patente?}

Produtos ou processos tangíveis, que podem ser produzidos ou utilizados de forma concreta e ter um grau de inovação.

\section{Por que um software independente não pode ser patenteado, e um software embarcado pode?}

Um software é uma implementação de uma funcionalidade, ele é protegido pela lei de direito autoral, mas não por patente, pois ele não é tangível, não é produzido. 
Ja um software embarcado é parte de um produto concreto, que entra em linha de produção, assim ele pode ser patenteado como parte do produto.

\section{Por que uma patente pode ser uma vantagem competitiva?}

Pois garante que somente sua empresa poderá explorar um produto ou processo de produção durante um determinado período de tempo, eliminando a concorrência na venda, ou possivelmente barateando o custo de produção no caso de um processo de produção, mas esse menor custo de produção só será alcançado pela empresa que detêm a patente.

\section{Por que a patente protege o investimento em novas tecnologias? E como ela ajuda o avanço da ciência?}

Se alguém investe em desenvolvimento e inovação, e ao conseguir um produto, todas as outras empresas a copiam, somente uma empresa gastara com o desenvolvimento e todos usufruirão do beneficio, assim, garantir o direito de exploração de uma inovação, a partir de uma patente, garante que a empresa que inovou tenha retorno do investimento no desenvolvimento, antes de todos podem explorar o benefício.


\section{Quais são os tipos de marcas?}

\begin{itemize}

\item nominativas: Constituída de palavra.
\item figurativas: Parte visual, desenhos, imagens.
\item mistas: Combinação dos itens anteriores.
\item tridimensionais: Formato especial, como de uma embalagem como uma garrafa de refrigerante.

\end{itemize}


\section{Em que circunstâncias é possível duas empresas distintas registrarem marcas idênticas?}

Quando a mesma marca está relacionada a classes de produtos diferentes, como uma nome para ser explorado por uma empresa de máquinas e outra de produtos alimentícios.

\section{O que é patenteável? Quais os critérios fundamentais?}

Patentes são aplicáveis a invenções concretas, passíveis de aplicação industrial. É preciso também haver novidade e atividade inventiva.

\section{O regime de proteção ao software é o direito autoral, e o direito autoral não protege ideias em si, apenas a forma com que são expressas. Tendo isso em vista, quais porções do software são passíveis de proteção (e quais porções não são passíveis de proteção)?}

O código em si é passível de proteção de direito autoral. No entanto a ideia de um problema, uma funcionalidade, bem como a solução para esses casos, não são protegidos pelo direito autoral.

\end{document}
