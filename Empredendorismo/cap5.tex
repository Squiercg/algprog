\documentclass[a4paper,10pt]{article}
\usepackage[utf8]{inputenc}
\usepackage[brazil]{babel}
\usepackage[T1]{fontenc}

\usepackage[cm]{fullpage}

\author{Empreendedorismo - Augusto Cesar de Aquino Ribas}
\title{Exercício em sala - Capítulo 5}


\begin{document}

\maketitle


\section{Por que o marketing não equivale exclusivamente a esforço de vendas? O que, além de vendas, inclui o marketing?}

O marketing não compreende apenas o esforço de venda, mas também entender o consumidor, quem é esse consumidor e seu perfil, quantas pessoas são consumidores em potencial para o seu produto, e se seu produto realmente é uma necessidade para elas. Assim desde o momento do design, o marketing ja está envolvido. Estudar o mercado.

\section{O que quer dizer a expressão: "o produto se vende sozinho" (Peter Druker 1 )?}

O produto 


\section{Quais são os 4 Ps do marketing? Qual o significado de cada um?}

\begin{itemize}
\item produto : que satisfaça o cliente, que 
\item preço : acessivel 
\item praça
\item promoção

\end{itemize}


\section{O que é segmentação? Cite exemplos de critérios de segmentação.}

\section{Em um plano de negócios, quais perguntas fundamentais devem ser respondidas com respeito ao Conceito do Negócio? E com relação a Análise de Mercado? E com relação ao Plano de Marketing?}

\end{document}
